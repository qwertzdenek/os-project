% ==========================
% ŠABLONA XETEX ČLÁNEK
% ==========================


\documentclass[a4paper,12pt]{article}

% ==========================
%	Balíčky
% ==========================
%\usepackage[czech]{babel}		% nastavení češtiny (dělení slov, odstavce, nadpisy, datumy...)
%\usepackage[utf8x]{inputenc}		% nastavení vstupního kódování na UTF8
%\usepackage[T1]{fontenc}		% kodování fontu pro výsledný dokument

\usepackage{listings}
\lstset{language=C++, numbers=left, backgroundcolor=\color{Azure}}

% podpora jazyků
\usepackage{polyglossia}

% pro použití obrázků
%\usepackage[pdftex]{graphicx}			

% pro podporu výpisů programu
\usepackage{listings}	

\usepackage{hyperref}

% pro definování vlastních barev - lepší než color
\usepackage[svgnames]{xcolor}						

% pro více obrázků u sebe s jednou popiskou
\usepackage{subcaption}		

% matematika vlepší a abstraktnější verzi
\usepackage{amsmath}

% ==========================
%	Nastavení
% ==========================			

\graphicspath{ {./obrazky/} }

% počeštění názvu výpisu kódů
\renewcommand{\lstlistingname}{Kód}



\definecolor{bluekeywords}{rgb}{0.13,0.13,1}
\definecolor{greencomments}{rgb}{0,0.5,0}
\definecolor{redstrings}{rgb}{0.9,0,0}


\lstdefinestyle{sharpc}{
	language=[Sharp]C, 
	tabsize=4,
	keywordstyle=\color{blue}}

\lstset{style=sharpc}



% vycentrování obsahu všech floatů
\makeatletter
\g@addto@macro\@floatboxreset\centering
\makeatother

% vlastní dělení slov
\hyphenation{vy-ge-ne-ro-va-ných}


% ==============================
%	Začátek dokumentu
% ==============================
\begin{document}

% ==============================
%	Titulní strana
% ==============================
\begin{titlepage}

\sffamily	% odtud se píše bezpatkovým písmem

	\begin{center}
		\begin{Large}
		
		Západočeská univerzita v~Plzni

		\vspace*{0.2cm}
		
		Fakulta aplikovaných věd

		\vspace*{0.2cm}
		
		Katedra informatiky a výpočetní techniky
		
		\vspace*{5mm}

		% vložení loga ZČU	
		\includegraphics[width=0.25\textwidth]{obrazky/logo_zcu}	
		
		\vspace*{2cm}
		
		% Název předmětu
		{\Huge\bfseries Semestrální práce z~předmětu Operační systémy}

		\vspace*{1cm}
		
%		% Název úlohy
		{\bfseries SMP}
		\end{Large}
	\end{center}
	
	% vyplní mezerami do konce stránky
	\vfill

	% vloží čáru o tloušťce 0,4 pt (tloušťka kartonu)
	\hrule
	
	\vspace*{0.2cm}	
	
	% neodsazovat první řádku kontaktu
	\noindent
	Zdeněk Janeček \\ 
	david.fido.fiedler@gmail.com \\
	A14N0111P \\
	Plzeň, \number\day. \number\month. \number\year

	\vspace*{0.2cm}	
	
	\noindent
	David Fiedler \\ 
	david.fido.fiedler@gmail.com \\
	A14N0111P \\
	Plzeň, \number\day. \number\month. \number\year
	
	\vspace*{0.2cm}	
	
	\noindent
	Tomáš Cígler \\ 
	david.fido.fiedler@gmail.com \\
	A14N0111P \\
	Plzeň, \number\day. \number\month. \number\year

\rmfamily	% odtud se opět píše patkovým písmem

\end{titlepage}


\section{Zadání}

\section{Řešení}

\subsection{Virtuální SMP}
Spouštění procesoru probíhá obdobně jako u~reálného procesoru. Vstupem je procedura
\verb+hardware_start+. Nejdříve se inicializuje
systém obsluh přerušení. Každé přerušení je mapováno na samostatný Handler
hostitelského systému. Použil jsem úplně základní Event. Ten je v~počátečním stavu
nesignalizovaný. Každé přerušení lze také zamaskovat a má svojí obslužnou rutinu.

V~naší aplikaci jsou k~dispozici následují přerušení.

\begin{description}
\item{INT\_SCHEDULER} Přepnutí plánovače. Běží typicky na prvním procesoru. Ostatní
jádra mají toto přerušení zamaskované.
\item{INT\_RESCHEDULE} Přepnutí úlohy na procesoru. Přerušení očekává adresu
zásobníku v~příslušné zprávě.
\item{INT\_CORE\_TERM} Ukončení jádra.
\item{INT\_CORE\_RESUME} Opětovné spuštění. Používá plánovač.
\item{INT\_CORE\_SUSPEND} Pozastavení jádra.
\end{description}

Prvním vzniklým vláknem je vlákno hardwarového přerušení, které slouží k~signalizaci
plánovače. Stará se také o~životní cyklus celé simulace SMP. Toto vlákno simuluje
fyzické zapojení a tedy běží s~nejvyšší prioritou.

Každé jádro má svoje vlákno obsluhy přerušení. To odpovídá hodinám, kde nebyly třeba
ve své původní podobě, protože instrukce obslužného jádra běží nezávisle na hodinách
jako nativní kód. Při inicializaci vlákna obsluhy přerušení vznikne, také fyzické jádro.

Dále se inicializuje plánovač. To je také první úloha, která se spustí na prvním procesoru.
Plánovač si vytvoří základní kontext, ze kterého pak vznikají další odvozené. Mimo inicializace
paměti vytvoří první přerušení plánovače. Plánovač zavede první jádro a začně provádět
svůj kód. Více o~plánovači v~další části.

Vypnutí procesoru zajistí procedura \verb+power_button+. Ve skutečnosti vznikne event,
na který obslužné vlákno hardwarového přerušení zareaguje a rozešle sadu přerušení pro
ukončení všech jader. Jakmile je procesor ukončen, provede ukončení všech otevřených
handlerů přerušení.

\subsection{Obsluha přerušení}
V~předchozí sekci jsem se již zmínil o~technickém provedení přerušení a to systémové
události, na která čekám pomocí \verb+WaitForMultipleObjects+. Některá přerušení mají
obsluhu, která běží na cílovém procesoru, ale některá slouží k~ovládání jádra.
Z~těch co běží na jádře jsme si vystačili s~obsluhou plánovače a přeplánování.

Přerušení plánovače \verb+INT_SCHEDULER+ nejdříve získá vnitřní zámek plánovače aby
nedocházelo k~souběhu s~některým systémovým voláním. Pak pozastaví všechna jádra a uloží jejich kontext. Jádru je tím předložena instrukce obsluhy, která použije svůj zásobník plánovače.
Plánovač pak vrazí přes registr EAX adresu následujícího zásobníku. Ten je načten běžnou cestou.
Návrat do přerušené úlohy je přes instrukci RET. Na prvním procesoru vždy něco běží.

Díky tomu je možné přeskládat úlohy s~jejich aktuálním stavem. Kontext na zásobníku
obsahuje návratovou adresu tj. následující instrukce úlohy, stavové vlajky a pak všechny
registry v~pořadí jak je vyžaduje instrukce \texttt{popad}.

Přerušení přeplánování \verb+INT_RESCHEDULE+ pozastaví cílové jádro a počká na zámek
plánovače aby mohlo uložit celý kontext aktuálně běžícího vlákna. Z~matice zpáv si přečte
cílový ESP, který vloží do kontextu jádra. Stejně tak jako v~předchozím případě. Vloží
do EIP adresu obsluhy přeplánování. Cílové jádro je probuzeno.

Obsluha přerušení je napsána v~Assembleru, aby byl zaručen minimální overhead. Využívám tak
instrukce \texttt{popad} pro načtení obecných registrů a tak  \texttt{popfd} pro načtení vlajek.

\subsection{Plánovač}
V~těle plánovače dojde nejdříve k~ověření zda nebylo pozastaveno nějaké jádro. To je
způsobeno zavoláním voláním plánovače \verb+sched_request_resume+ a \verb+sched_request_pause+.
Jakmile něco běží na pozastaveném procesoru, musí se to ukončit a uložit kontext.
Tato úloha pak putuje na konec fronty úloh.

Dále se mažou všechny úlohy které ukončily své tělo. Struktura TCB obsahuje destruktor,
takže jsou v~tu chvíly uvolněny veškeré alokované prostředky.

Ještě před spuštěním samotného plánovaní jsou načteny požadavky na nové úlohy. To zahrnuje
vytvoření nového Task Control Block (TCB). O~inicializaci dat se stará funkce
\verb+sched_create_task+. Zásobník musí být souvislý a zarovnaný. Vložíme do něj postupně:

\begin{itemize}
\item argument úlohy
\item návratová adresa do endtask callbacku
\item vstupní adresa úlohy
\item vlajky
\item obecné registry
\end{itemize}

Na hodnotách registrů nezáleží. Vkládám tam nějaké hodnoty abych poznal že
se vše načítá správně. Dále je nastaven Task ID (TID), časové kvantum, stav Runnable,
typ úlohy a také předávaná struktura, kvůli dealokaci paměti.

Činnost plánovače je pro každé jádro stejná:

\begin{itemize}
\item ověříme zda je jádro povolené
\item podíváme se do tabulky spuštěných procesů na proces co na jádru běží. Jestliže tam nic neběží
pokračuje v~další části.
\begin{itemize}
\item ubereme mu časové kvantum TIME\_QUANTUM\_DECREASE
\item pokud už běžící proces vyčerpal čas TIME\_QUANTUM, zařadí se do fronty procesů.
IDLE task v~tu chvíli končí. Pokud kvantum nevypršelo je navrácen zpět mezi naplánované úlohy a pokračuje
dalším jádrem.
\end{itemize}
\item podle fronty úloh
\begin{itemize}
\item Je-li prázdná a jsme na prvním procesoru, je spuštěn IDLE task. Jsme-li na jiném jádře. Je to
jádro pozastaveno.
\item máme-li čekající úlohu ve frontě, je vybrána jako následující
\end{itemize}
\item pokud je vybraná úloha ta stejná která tam byla před přeplánováním, je tato změna ignorována
\item Nakonec proběhne samotné přerušení cílových jader a odeslání cílového ESP. První jádro je
přeplánováno při návratu z~plánovače.
\end{itemize}

\subsection{Životní cyklus úlohy}
Nyní popíšem životní cyklus aplikace od jejího vzniku, stavy běhu a nakonec její
ukončení. Naše schéma je na obrázku \ref{fig:state_diagram}. Z~běžného schématu
vypadl stav blokovaný. Protože používáme aktivní čekání. Navíc nemám k~dispozici
přerušení na úrovni instrukcí jen signalizaci. Musel bych jádro ručně zastavovat a
to by mělo za následek neočekávané chování.

\begin{figure}
\centering
\includegraphics[width=\textwidth]{obrazky/state_diagram.png}
\caption{Stavový diagram úloh}
\label{fig:state_diagram}
\end{figure}

Každý task může spouštět další úlohy a to pomocí volání \verb+exec_task()+. První
parametr je typ a další je ukazatel na strukturu parametrů. Záleží na typu úlohy a
může být NULL. Tento ukazatel může být sdílen mezi mnoha úlohy. Proto je také
typován jako \verb+shared_ptr+ aby byl po skončení uklizen. Není hlídána cyklická
reference.

Jakmile se naplánuje plánovač projde frontu nových úloh, vytvoří jejich TCB a zařadí
do fronty čekajících úloh v~\verb+task_queue+.

Jakmile se na ní dostane řada, přejde do stavu RUNNING. Plánovač zajistí přepnutí úlohy.

Každá úloha se může nacházet jen v~jedné z~těchto front. Proto je také TCB unikátní pointer.
Jeho prostředky jsou tedy automaticky uvolněny po vymazání z~fronty
\verb+exit_task_queue+. Do té se dostane při návratu z~hlavní procedury v~metodě
\verb+end_task_callback()+. Ta je vložena do zásobníku při vytváření a tedy instrukce
RET do ní skočí.

\subsection{Systémová volání}
Každé systémové volání je definováno v~modulu \verb+sched_calls+. K~dispozici
je již zmíněné \verb+exec_task()+ a také \verb+get_tid()+. Tato volání jsou přímo
mapována na volání plánovače. To je z~důvodu odstínění úloh od plánovače. Tyto
volání jsou přesto volána synchronně a jsou co nejkratší.

Jednotlivé úlohy získávají parametry ze struktur libovolného obsahu. Jmenovitě
jsou to struktury:

\begin{lstlisting}
struct task_common_pointers {
	semaphore_t full;
	semaphore_t empty;
	semaphore_t mutex;

	circular_buffer buffer;

	volatile bool can_run;
	double mean;
	double deviation;
};
\end{lstlisting}

Ta slouží pro nastavení úloh PRODUCENT a CONSUMENT. Pro spuštění RUNNER je třeba naplnit:

\begin{lstlisting}
struct task_run_parameters {
	double mean;
	double deviation;
};
\end{lstlisting}

\subsection{Synchronizační primitiva}
Pro potřeby jak úloh, tak vnitřních struktur bylo potřeba implementovat vlastní
semafor. Knihovní funkce totiž nefungují. Z~pohledu systému máme pouze 4 vlákna
procesoru, ale ve skutečnosti synchronizujeme mezi neexistujícími úlohami.

Definovali jsme tedy strukturu \texttt{semaphore\_t}, která uchovává aktuální hodnotu
semaforu. Vzhledem k~tomu že jde o~celočíselnou hodnotu, je přiřazení atomické (pokud je
zaručeno zarovnání). Pro jistotu jsem ještě přidal prefix volatile.

Zkoušel jsem použít C++11 atomic volání, ale to jsem opustil, protože docházelo
k~nevysvětlitelnému chování \verb+__RTC_CheckEsp+, které je vkládáno. Tělo takového
zámku je v~následujícím výpisu. Nechávám zde pro snažší pochopení Assemblerové
implementace.

\begin{lstlisting}
int semaphore_P(semaphore_t &s, int value)
{
	int expected;
	int old;

	do {
		do {
			old = s._value;
			expected = old - value;
		} while (expected < 0);
	} while (!s._value.compare_exchange_weak(
		old, expected,
		std::memory_order_release,
		std::memory_order_relaxed));

	return expected;
}
\end{lstlisting}

Kód semaforu byl použit následující (\texttt{synchro.cpp}):

\lstset{language=[x86masm]{Assembler}}

\begin{lstlisting}
spin:
	mov edx,
s~mov edx, [edx]s._value
	mov ebx, edx
	sub ebx, value
	js spin

	mov eax, edx
	; eax - old, actual [s], ebx - expected
	mov ecx,
s~lock cmpxchg[ecx]s._value, ebx
	jnz spin
	mov result, ebx
\end{lstlisting}

\lstset{language=C++}

Vzhledem k~tomu že jsem cílil na x86, mohl jsem použít inline assembly
a obalit standardní funkcí.

Uvolnění hodnoty semaforu je jednoduché:

\begin{lstlisting}
void semaphore_V(semaphore_t &s, int value)
{
	s._value += value;
}
\end{lstlisting}

\section{Vzorové úlohy}

\section{Závěr}
Při vypracování této úlohy jsme narazili na mnoho problémů, které je třeba
řešit a prozkoumali jsme základní úlohu operačního systému a to sdílení réalného času.
Využili jsme jak vysokoúrovňového C++, tak Assembleru. Zaměřili jsme se na starší
archetekturu Intel x86, protože nevyužívá velké množství registrů než 64 bitová
následovník. Z~tohoto důvodu zmizely zásadní instrukce \texttt{popad} a \texttt{popfd},
které přečtou zásobník a přepíší registry. Na reálném procesoru je obsluha přerušení řešena
hardwarově a proto je možné obsluhovat s~takovou rychlostí.

Bylo třeba dát si pozor co skutečně překladač vytvoří. Narazil jsem na problém, když jsem
měl nastavený malý zásobník a překladač tiše předpokládal větší. Pak nastalo k~překryvu
adresního prostoru zásobníku a datového segmentu. Překladač totiž zahrnuje do kódu různé
čištění paměti pomocí instrukce \verb+rep stos+. Všechny takové chyby se vždycky projeví
později na nečekaném místě.

Ukázalo se že plánovač Round Robin není skutečně přiliš optimální a bez řešení stavu
blocked, je čas obrátky velmi dlouhý. Často se tak stane, že běží na procesoru jen jeden
z~dvojice producent a konzument a tedy nedělá nic jiného než čeká. Plánovač by musel
sledovat kdo je kým blokovaný a dávat je tedy k~sobě například prioritou.

Tato práce byla velmi poučná, ale také časově náročná.


\begin{thebibliography}{9}

\bibitem[1]{Wikibooks}
{\em Wikipedia contributors} \\
{\bf "LaTex" Wikibooks - open book for an open world} \\
\url{http://en.wikibooks.org/wiki/LaTeX} \\

\end{thebibliography}

\end{document} 

