% ==========================
% ŠABLONA XETEX ČLÁNEK
% ==========================


\documentclass[a4paper,12pt]{article}

% ==========================
%	Balíčky
% ==========================
%\usepackage[czech]{babel}		% nastavení češtiny (dělení slov, odstavce, nadpisy, datumy...)
%\usepackage[utf8x]{inputenc}		% nastavení vstupního kódování na UTF8
%\usepackage[T1]{fontenc}		% kodování fontu pro výsledný dokument

% podpora jazyků
\usepackage{polyglossia}

% pro použití obrázků
%\usepackage[pdftex]{graphicx}			

% pro podporu výpisů programu
\usepackage{listings}	

\usepackage{hyperref}

% pro definování vlastních barev - lepší než color
\usepackage{xcolor}						

% pro více obrázků u sebe s jednou popiskou
\usepackage{subcaption}		

% matematika vlepší a abstraktnější verzi
\usepackage{amsmath}

% ==========================
%	Nastavení
% ==========================			

\graphicspath{ {./obrazky/} }

% počeštění názvu výpisu kódů
\renewcommand{\lstlistingname}{Kód}



\definecolor{bluekeywords}{rgb}{0.13,0.13,1}
\definecolor{greencomments}{rgb}{0,0.5,0}
\definecolor{redstrings}{rgb}{0.9,0,0}


\lstdefinestyle{sharpc}{
	language=[Sharp]C, 
	tabsize=4,
	keywordstyle=\color{blue}}

\lstset{style=sharpc}



% vycentrování obsahu všech floatů
\makeatletter
\g@addto@macro\@floatboxreset\centering
\makeatother

% vlastní dělení slov
\hyphenation{vy-ge-ne-ro-va-ných}


% ==============================
%	Začátek dokumentu
% ==============================
\begin{document}

% ==============================
%	Titulní strana
% ==============================
\begin{titlepage}

\sffamily	% odtud se píše bezpatkovým písmem

	\begin{center}
		\begin{Large}
		
		Západočeská univerzita v Plzni

		\vspace*{0.2cm}
		
		Fakulta aplikovaných věd

		\vspace*{0.2cm}
		
		Katedra informatiky a výpočetní techniky
		
		\vspace*{5mm}

		% vložení loga ZČU	
		\includegraphics[width=0.25\textwidth]{obrazky/logo_zcu}	
		
		\vspace*{2cm}
		
		% Název předmětu
		{\Huge\bfseries Semestrální práce z předmětu Operační systémy}

		\vspace*{1cm}
		
%		% Název úlohy
		{\bfseries SMP}
		\end{Large}
	\end{center}
	
	% vyplní mezerami do konce stránky
	\vfill

	% vloží čáru o tloušťce 0,4 pt (tloušťka kartonu)
	\hrule
	
	\vspace*{0.2cm}	
	
	% neodsazovat první řádku kontaktu
	\noindent
	Zdeněk Janeček \\ 
	david.fido.fiedler@gmail.com \\
	A14N0111P \\
	Plzeň, \number\day. \number\month. \number\year

	\vspace*{0.2cm}	
	
	\noindent
	David Fiedler \\ 
	david.fido.fiedler@gmail.com \\
	A14N0111P \\
	Plzeň, \number\day. \number\month. \number\year
	
	\vspace*{0.2cm}	
	
	\noindent
	Tomáš Cígler \\ 
	david.fido.fiedler@gmail.com \\
	A14N0111P \\
	Plzeň, \number\day. \number\month. \number\year

\rmfamily	% odtud se opět píše patkovým písmem

\end{titlepage}


\section{Zadání}

\section{Návrh}

\section{Řešení}

\section{Závěr}
Zadání se podařilo splnit. 

\begin{thebibliography}{9}

\bibitem[1]{Wikibooks}
{\em Wikipedia contributors} \\
{\bf "LaTex" Wikibooks - open book for an open world} \\
\url{http://en.wikibooks.org/wiki/LaTeX} \\

\end{thebibliography}

\end{document} 

